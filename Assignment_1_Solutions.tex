\RequirePackage[l2tabu, orthodox]{nag}

\documentclass{article}
\usepackage{fullpage}
\usepackage{oz}
\usepackage{definitions}
\usepackage{todonotes}

\title{\bf Assignment 1: Background theory}
\author{Maxwell Bo  ~~ 4392687}

\begin{document}
\maketitle

\begin{enumerate}
    \item 
        \begin{enumerate}
            \item \[ y: [\ \true,\ (x = 0 \land y = 0) \lor (y = y_0 / x \land x \neq 0)\ ]\]
            \item \[ y: [\ \true,\ (x = 0) \lor (y = y_0 / x \land x \neq 0)\ ]\]
            \item TODO
        \end{enumerate}
    \item 
        \DERIVE
        \form{x,y:[\true, x = z^2 \land y = z^4]}
        \hint{\refsto} {Composition Rule}
        \form{x,y:[\true, x = z^2]; x,y:[x = z^2, x = z^2 \land y = z^4]}
        \hint{\refsto} {Assignment Rule: $\true \entails x = z^2 [x \backslash z^2]$}
        \form{x = z^2; x,y:[x = z^2, x = z^2 \land y = z^4]}
        \hint{\refsto} {Assignment Rule: $x = z^2 \entails x = z^2 \land y = z ^ 4[y \backslash x^2]$}
        \form{x := z^2; y := x^2} 
        \ENDDERIVE
    \item
        \begin{enumerate}
            \item Assuming
                \begin{eqnarray*}
                    wp(y := 10,\ \true) & \equiv & \true[y \backslash 10]\\
                                       & \equiv & \true
                \end{eqnarray*} 
                    we can conclude that 
                \begin{eqnarray*}
                    wp(\If~ (x > 0 ~\lor~ y < 10) \rightarrow y:= 10 ~\Fi,\ \true) & \equiv & (x > 0 ~\lor~ y < 10) ~\land~ \\
                        & & ((x > 0 ~\lor~ y < 10) \rightarrow wp(y := 10,\ \true))\\
                    & \equiv & (x > 0 ~\lor~ y < 10) ~\land~ \true\\
                    & \equiv & (x > 0 ~\lor~ y < 10)
                \end{eqnarray*}
                As $y < 10 \entails (x > 0 \lor y < 10)$, the Hoare triple is $\true$.
            \item Assuming
                \begin{eqnarray*}
                    wp(x := x + y,\ P[x \backslash x + y]) & \equiv & (P[x \backslash x + y])[x \backslash x + y]\\
                \end{eqnarray*}
        \end{enumerate}
    \item TODO
    \item TODO
\end{enumerate}

Given 

\[max(A,l,h,i) ~\defs~ \all j \in [l,h) \cdot (A_j \leq A_i) \land (l \leq i < h)\]   % \[ and \] tell Latex we are entering and leaving math mode

suppose we wanted to show that the specification

\[i,j:[A.len > 0, max(A,0,A.len,i)]\]

is refined by

\[i,j:=0,1;\\  % \\ adds a line break
\begin{array}{l}
\Do~ j < A.len \rightarrow\\
\t1 \begin{array}{l}   % \t1 introduces a tab. The array environment allows us to write 
\If~ A_j > A_i \rightarrow i:=j\\
\Choice  A_j \leq A_i \rightarrow {\bf skip} \\
\Fi; \\
j:=j+1\\
\end{array}\\
\Od\\
\end{array}\]

Parts of the proof follow:

\DERIVE
\form{i,j:[A.len > 0, max(A,0,A.len,i)]}
\hint{\refsto} {Strengthen post: $inv \land max(A,0,A.len,i) \entails max(A,0,A.len,i)$}  % $ changes between text and math mode
\form{i,j:[A.len > 0, inv \land max(A,0,A.len,i)]}
\hint{\equiv}{$max(A,0,A.len,i)$ is equivalent to $j = A.len$ when $inv$ is true}
\form{i,j:[A.len > 0, inv \land j  = A.len]}
\hint {\refsto} {Composition: mid predicate is $inv$}
\form {i,j:[A.len > 0, inv]; i,j:[inv, inv \land j= A.len]}
\hint {\refsto} {Assignment: $A.len > 0 \entails inv[i,j\backslash 0,1]$}
\form { i,j:=0,1; i,j:[inv, inv \land j= A.len]}
\ENDDERIVE
  
The proof of the final step above is:

\DERIVE
\form{inv[i,j\backslash 0,1]}
\hint{\equiv}{definition of $inv$}
\form{max(A,0,j,i)[i,j\backslash 0,1]}
\hint{\equiv}{apply substitution}
\form{max(A,0,1,0)}
\hint{\equiv}{since $A_0$ only element in $A_{[0,1)}$}
\form{\true}
\ENDDERIVE

\pagebreak % forces a page break here

Continuing the refinement:

\DERIVE
\form{i,j:[inv, inv \land j=A.len]}
\hint{\refsto}{Repetition: $A.len-j$ is variant}
\form{\begin{array}{l}   
\Do~ j\neq A.len \rightarrow\\   
\t1 i,j:[inv \land j< A.len, inv \land (0 \leq A.len-j < A.len-j_0)]\\  
\Od \\
\end{array}}
\ENDDERIVE

Here is another part of the proof involving other GCL notation:

\DERIVE
\form{i:[inv \land j< A.len, max(A,0,j+1,i)]}
\hint{\refsto}{Selection: $P \entails A_i > A_j \lor A_j \leq A_i$, for any $P$}
\form{\begin{array}{l}
\If~ A_j > A_i \rightarrow i:[A_j > A_i \land inv \land j< A.len, max(A,0,j+1,i)]\\
\Choice A_j \leq A_i \rightarrow i:[A_j \leq A_i \land inv \land j< A.len, max(A,0,j+1,i)]\\
\Fi\\
\end{array}}
\hint{\sqsubseteq}{Assignment: $A_j > A_i \land inv \land j< A.len \entails max(A,0,j+1,i)[i\backslash j]$}
\form{\begin{array}{l}
\If~  A_j > A_i \rightarrow i:=j\\
\Choice A_j \leq A_i \rightarrow i:[A_j \leq A_i \land inv \land j< A.len, max(A,0,j+1,i)]\\
\Fi\\
\end{array}}
\hint{\refsto}{Skip: $A_j \leq A_i \land inv \land j< A.len \entails  max(A,0,j+1,i)$}
\form{\begin{array}{l}
\If~ A_j > A_i \rightarrow i:=j\\
\Choice A_j \leq A_i \rightarrow {\Skip} \\
\Fi\
\end{array}}
\ENDDERIVE
  
\end{document}

