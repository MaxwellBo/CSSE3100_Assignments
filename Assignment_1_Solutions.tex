% \RequirePackage[l2tabu, orthodox]{nag}

\documentclass{article}
\usepackage{fullpage}
\usepackage{oz}
\usepackage{definitions}
\usepackage{todonotes}

\title{\bf Assignment 1: Background theory}
\author{Maxwell Bo  ~~ 43926871}

\begin{document}
\maketitle

\begin{enumerate}
\item 
    \begin{enumerate}
    \item $y: [\ \true,\ (x = 0 \Rightarrow y = 0) \land (x \neq 0  \Rightarrow y = \frac{y_0}{x})\ ]$
    \item $y: [\ \true,\ (x = 0 \Rightarrow \true) \land (x \neq 0  \Rightarrow y = \frac{y_0}{x})\ ]$
    \item Let $S = [\ x \neq 0,\ y = \frac{y_0}{x}\ ]$, $A =$ (a) and $B = $ (b).\\ 
    $B$ refines to $A$, as $A$ has a stronger postcondition than that of $B$. The user may replace a program written to the specification of $B$ with one written to $A$ and uphold the contract. $S$ does not refine to $B$. While the precondition of $B$ is weaker than that of $S$, the postcondition of $B$ is not stronger or equivalent to that of $A$. As $S$ does not refine to $B$, $S$ does not refine to $A$, by the law of transitivity. 
    \end{enumerate}

\item 
    \DERIVE
    \form{x,y:[\true,\ x = z^2 \land y = z^4]}
    \hint{\refsto} {Composition}
    \form{x,y:[\true,\ x = z^2]; x,y:[x = z^2,\ x = z^2 \land y = z^4]}
    \hint{\refsto} {Assignment: $\true \entails x = z^2 [x \backslash z^2]$}
    \form{x = z^2; x,y:[x = z^2,\ x = z^2 \land y = z^4]}
    \hint{\refsto} {Assignment: $x = z^2 \entails (x = z^2 \land y = z ^ 4)[y \backslash x^2]$}
    \form{x := z^2; y := x^2} 
    \ENDDERIVE

\item
    \begin{enumerate}
    \item Assuming
        \begin{eqnarray*}
            wp(y := 10,\ \true) & \equiv & \true[y \backslash 10]\\
                               & \equiv & \true
        \end{eqnarray*} 

        we can conclude that 

        \begin{eqnarray*}
            wp(\If~ (x > 0 ~\lor~ y < 10) \rightarrow y:= 10\ \Fi,\ \true) & \equiv & (x > 0 ~\lor~ y < 10) ~\land~ \\
                & & ((x > 0 ~\lor~ y < 10) \rightarrow wp(y := 10,\ \true))\\
            & \equiv & (x > 0 ~\lor~ y < 10) ~\land~ \true\\
            & \equiv & (x > 0 ~\lor~ y < 10)
        \end{eqnarray*}

        As $y < 10 \entails (x > 0 \lor y < 10)$, the Hoare Triple is $\true$.

    \item
        \begin{eqnarray*}
            wp(x := x + y,\ P[x \backslash x + y]) & \equiv & (P[x \backslash x + y])[x \backslash x + y]\\
                & \equiv & P[x \backslash (x + y) + y]\\
                & \equiv & P[x \backslash x + 2y]
        \end{eqnarray*}

        Assuming that $P$ is quantified over all possible predicates,        

        \begin{equation}
        \hoare{P}{x := x + y}{P[x \backslash x + y]}
        \end{equation}
         
        does not hold, due to the choice of $P$ determining the validity of the Hoare Triple.\\

        As a counter example, let $P \equiv (x = 0)$, such that the Triple is

        \begin{equation}
        \hoare{x = 0}{x := x + y}{x = 0[x \backslash x + y]}
        \end{equation}

        Given that

        \begin{eqnarray*}
            wp(x := x + y,\ x = 0[x \backslash x + y]) & \equiv & x = 0[x \backslash x + 2y]\\
                & \equiv & x + 2y = 0
        \end{eqnarray*}

        \begin{equation}
        (x = 0) \not\entails (x + 2y = 0)\\
        \end{equation}

        \begin{equation}
        \therefore \not\forall P : (\hoare{P}{S}{Q}) \rightarrow (P \entails wp(S, Q))\\
        \end{equation}

        Thus, the Hoare Triple is $\false$.

    \end{enumerate}
\item
    \begin{enumerate}
    \item 
        \DERIVE
        \form{y:[y < 10, y > 0]}
        \hint{\refsto} {Selection: $y < 10 \entails (x > 0 \lor y < 10)$}
        \form{\If~ (x > 0 \lor y < 10) \rightarrow y:[(x > 0 \lor y < 10) \land (y < 10),\ y > 0]\ \Fi}
        \hint{\refsto} {Absorption 1: $(x > 0 \lor y < 10) \land (y < 10) = y < 10$}
        \form{\If~ (x > 0 \lor y < 10) \rightarrow y:[y < 10,\ y > 0]\ \Fi}
        \hint{\refsto} {Assignment: $y < 10 \entails y > 0[y \backslash 10]$}
        \form{\If~ (x > 0 \lor y < 10) \rightarrow y := 10\ \Fi}
        \ENDDERIVE

    \item 
        \DERIVE
        \form{y:[y < 10, y > 0]}
        \hint{\doesntrefsto} {Selection: $y < 10 \not\entails ((x > 0) \land (y < 10))$}

        \form{\If~ ((x > 0) \land (y < 10)) \rightarrow y:[((x > 0) \land (y < 10)) \land (y < 10),\ y > 0]\ \Fi}
        \ENDDERIVE

        The precondition cannot strengthened. Thus, we cannot refine to a selection statement using $((x > 0) \land (y < 10))$ as its only guard.
    \end{enumerate}
\item

    \[\forall S,\ B : (\Repeat\ S\ \Until\ B ~\equiv~ S; \Do\ \neg B \rightarrow S\ \Od)\]

    $\therefore$

    \DERIVE
    \form{w:[P, Q]}
    \hint{\refsto} {Composition}
    \form{w: [P, I]; w: [I, Q]}
    \hint{\refsto} {Strengthen Postcondition: $I \land \neg(\neg B) \entails Q$}
    \form{w: [P, I]; w: [I, I \land \neg(\neg B)]}
    \hint{\refsto} {Repetition}
    \form{w: [P, I]; \Do~ (\neg B) \rightarrow
 w:[I \land \neg B,\ I \land (0 \leqslant V < V_0)]\ \Od}
    \ENDDERIVE

    $w: [P,\ I]$ and $w:[I \land \neg B,\ I \land (0 \leqslant V < V_0)]$ can both refine to the same program, $S$.\\

    Thus, $P \entails I \land \neg B$, such that for an arbitrary $S$ with precondition $P$, $S$ satisfies the requirements of the $\Do$ loop.\\

    Given both $I \land \neg(\neg B) \entails Q$ and $P \entails I \land \neg B$, we can formulate\\

    \begin{center}
    if $P \entails I \land \neg B$ then $w:[P,\ I \land B] ~\refsto~ \Repeat\ w:[I \land \neg B,\ I \land (0 \leqslant V < V_0)]\ \Until\ B$\\
    where $I$ is a loop invariant, and $V$ is a loop variant
    \end{center}

\end{enumerate}

\end{document}

