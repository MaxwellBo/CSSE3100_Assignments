% \RequirePackage[l2tabu, orthodox]{nag}

\documentclass[a4paper]{article}
\usepackage{fullpage}
\usepackage{oz}
\usepackage{definitions}
\usepackage{todonotes}
% \usepackage{amsmath}
% \usepackage{amssymb}

\newcommand{\LEN}{\mbox{len}}
\newcommand{\PRE}{lrun(A, n, n + 1)}
\newcommand{\POST}{mrun(A, n_0, m)}
\newcommand{\INV}{lrun(A, n_0, m)}
\newcommand{\ENV}{n, m}
\newcommand{\ASSIGNMENT}{m := n + 1}
\newcommand{\GUARD}{(m < A.\LEN \land A_{n_{0}} = A_m)}
\newcommand{\MRUN}[3]{lrun({#1}, {#2}, {#3}) \land ({#3} < {#1}.\LEN \Rightarrow {#1}_{#2} \neq {#1}_{#3})}
\newcommand{\IMPLIESEXPANSION}[2]{\neg ({#1}) \lor ({#2})}
\newcommand{\VARIANTEXP}{(0 \leqslant V < V_0)}


\newcommand{\PREP}{A.\LEN > 0}
\newcommand{\POSTP}{mrun(A, \ell, h) \land (\forall p, q \,\cdot\, mrun(A, p, q) \Rightarrow (h - \ell) \leqslant (q - p))}
\newcommand{\ENVP}{\ell, h}
\newcommand{\INVP}{\mbox{TODO}}



\title{\bf Assignment 3: Derivation}
\author{Maxwell Bo ~~ 43926871}

\begin{document}
\maketitle

% \DERIVE
% \form{inv[i,j\backslash 0,1]}
% \hint{\equiv}{definition of $inv$}
% \form{max(A,0,j,i)[i,j\backslash 0,1]}
% \hint{\equiv}{apply substitution}
% \form{max(A,0,1,0)}
% \hint{\equiv}{since $A_0$ only element in $A_{[0,1)}$}
% \form{\true}
% \ENDDERIVE


\begin{enumerate}
	\item 
	\begin{enumerate}
		\item $n$ is a \textbf{value} parameter. $m$ is a \textbf{result} parameter.
		\item $inv ~\triangleq~ \INV$
		\item 

Let

\begin{eqnarray*}
pre(A, n) & \triangleq & \PRE\\
inv(A, n, m) & \triangleq & \INV\\
post(A, n, m) & \triangleq & \POST
\end{eqnarray*}

by 1.(b), and the specification of the procedure. $inv$, $pre$ and $post$ implicitly capture variables $(A, n, m)$ as parameters from the frame, for syntactic convience.


s.t.

\DERIVE
\form{\ENV : [pre,\, post]}
\hint{\refsto} {Composition: middle predicate is $inv$}
\form{\ENV : [pre,\, inv];\ \ENV : [inv,\, post]}
\hint{\refsto} {Assignment: $pre \entails inv[m \backslash n + 1]$}
\form{\ASSIGNMENT;\ \ENV: [inv,\, post]}
\ENDDERIVE

$\because$

\begin{eqnarray*}
inv[m \backslash n + 1] & \equiv & \INV[m \backslash n + 1]\\
& \equiv & lrun(A, n_0, n + 1)
\end{eqnarray*}

$\therefore$ 

\begin{eqnarray*}
\PRE & \entails & lrun(A, n_0, n + 1)
\end{eqnarray*}


Let


\begin{eqnarray*}
guard & \triangleq & \GUARD
\end{eqnarray*}

s.t.

\DERIVE
\hint{\refsto} {Strengthen post: $inv \land \neg guard ~\entails~ post$}
\form{\ASSIGNMENT;\ \ENV: [inv,\, inv \land \neg guard]}
\ENDDERIVE

$\because$

\DERIVE
\form{inv \land \neg guard ~\entails~ post}
\hint{\equiv} {Expansion of definitions}
\form{\INV \land \neg \GUARD ~\entails~ \POST}
\hint{\equiv} {Expansion of functions}
\form{\INV \land \neg \GUARD ~\entails~ \MRUN{A}{n_0}{m}}
\hint{\equiv} {De Morgan's law - negation of conjunction}
\form{\INV \land (\neg (m < A.\LEN) \lor \neg (A_{n_{0}} = A_m)) ~ \entails~ \MRUN{A}{n_0}{m}}
\hint{\equiv} {$P \Rightarrow Q  ~\equiv~ \neg P \lor Q$}
\form{\INV \land (\neg (m < A.\LEN) \lor \neg (A_{n_{0}} = A_m)) ~ \entails~ lrun(A, n_0, m) \land (\IMPLIESEXPANSION{m < A.\LEN}{A_{n_{0}} \neq A_m})}
\hint{\equiv} {}
\form{\true}
\ENDDERIVE



\DERIVE
\hint{\refsto} {Repetition}
\form{
\begin{array}{l}
\ASSIGNMENT;\\
\Do~ \GUARD \rightarrow\\
    \t1 \ENV:[inv \land guard,\, inv \land \VARIANTEXP]\\
\Od\\
\end{array}}
\ENDDERIVE

where

\begin{eqnarray*}
V & \triangleq & A.\LEN ~-~ m
\end{eqnarray*}

\DERIVE
\hint{\refsto} {Assignment: $inv \land \neg guard ~\entails~ (inv \land \VARIANTEXP)[m \backslash m + 1]$}
\form{
\begin{array}{l}
\ASSIGNMENT;\\
\Do~ \GUARD \rightarrow\\
    \t1 m := m + 1\\
\Od\\
\end{array}}
\ENDDERIVE

$\because$

\begin{eqnarray*}
(inv \land \VARIANTEXP)[m \backslash m + 1] & \equiv & (\INV \land (0 \leqslant (A.\LEN - m) < (???)))[m \backslash m + 1]\\
 & \equiv & lrun(A, n_0, m + 1) \land (0 \leqslant (A.\LEN - (m + 1)) < (???))\\
\end{eqnarray*}

TODO

	\end{enumerate}

\item

\begin{eqnarray*}
pre(A) & \triangleq & \PREP\\
post(A, \ell, h) & \triangleq & \POSTP
\end{eqnarray*}

by 2. $pre$ and $post$ implicitly capture variables $(A, \ell, h)$ as parameters from the frame.


\DERIVE
\form{\ENVP : [pre,\, post]}
\hint{\refsto} {Composition: middle predicate is $inv$}
\form{\ENVP : [pre,\, inv];\ \ENVP : [inv,\, post]}
\ENDDERIVE

where

\begin{eqnarray*}
inv & \triangleq & \INVP
\end{eqnarray*}


where $inv$ implicitly captures varibles $(\mbox{TODO}$ as parameters from the frame.

\end{enumerate}


\end{document}
