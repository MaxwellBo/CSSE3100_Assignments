% \RequirePackage[l2tabu, orthodox]{nag}

\documentclass[a4paper]{article}
\usepackage{fullpage}
\usepackage{oz}
\usepackage{definitions}
\usepackage{todonotes}

\newcommand{\DOGUARD}[3]{({#1} \neq A.len\, \lor\, {#2} \neq B.len\, \lor\, {#3} \neq C.len)}
\newcommand{\ASSIGNMENT}{i, j, k, r := 0, 0, 0, 0}
\newcommand{\ENV}{i, j, k, r, D}
\newcommand{\POST}{D_{[0, r)} = A\, \cap\, B\, \cap\, C}
\newcommand{\INTERSECTION}[4]{D_{[0, {#1})} = A_{[0, {#2})}\, \cap\, B_{[0, {#3})}\, \cap\, C_{[0, {#4})}}
\newcommand{\BOUNDED}[4]{{#1} \in [0, D.len] ~\land~ {#2} \in [0, A.len] ~\land~ {#3} \in [0, B.len] ~\land~ {#4} \in [0, C.len]}
\newcommand{\BOUNDEDNONINCLUSIVE}[3]{{#1} \in [0, A.len) ~\land~ {#2} \in [0, B.len) ~\land~ {#3} \in [0, C.len)}
\newcommand{\VARIANTEXP}{(0 \leqslant V < V_0)}
\newcommand{\IFCASE}[1]{{#1} \rightarrow \ENV:[{#1} \land inv \land guard,\, inv \land \VARIANTEXP]}
\newcommand{\IFGUARD}{(A_i > B_j) \lor (B_j > C_k) \lor (C_k > A_i) \lor (A_i = B_j = C_k)}
\newcommand{\SUBSTITUTION}{[i, j, k, r \backslash A.len, 0, 0, 0]}


\title{\bf Assignment 2: Verification}
\author{Maxwell Bo ~~ 43926871}

\begin{document}
\maketitle

\section{Part A}

Let

\begin{eqnarray*}
pre & \triangleq & D.len \geqslant max(\{A.len,\, B.len,\, C.len\})\\
        && \land~ sorted(A) ~\land~ sorted(B) ~\land~ sorted(C)
\end{eqnarray*}

\begin{eqnarray*}
post(r) & \triangleq & \POST\\
\end{eqnarray*}


\DERIVE
\form{\ENV:[pre, post(r)]}
\hint{\refsto} {Composition: middle predicate is \textit{inv}}
\form{\ENV:[pre, inv]; \ENV:[inv, post(r)]}
\ENDDERIVE

where

\begin{eqnarray*}
inv & \triangleq & \INTERSECTION{r}{i}{j}{k}\\
        && \land~ \BOUNDED{r}{i}{j}{k}\\
\end{eqnarray*}
% HERE BE DRAGONS: Don't forgot that these ranges are noninclusive array slices

\DERIVE
\hint{\refsto} {Assignment: $pre \entails inv[i, j, k, r\backslash 0, 0, 0, 0]$}
\form{\ASSIGNMENT; \ENV:[inv, post(r)]}
\ENDDERIVE

$\because$

\begin{eqnarray*}
inv[i, j, k, r \backslash 0, 0, 0, 0] 
    & \equiv & \INTERSECTION{0}{0}{0}{0}\\
        && \land~ \BOUNDED{0}{0}{0}{0}\\
    & \equiv & \emptyset = (\emptyset\, \cap\, \emptyset\, \cap\, \emptyset) ~\land~ (\true ~\land~ \true ~\land~ \true ~\land~ \true)\\
    & \equiv & \emptyset = \emptyset ~\land~ \true\\
    & \equiv & \true
\end{eqnarray*}

\DERIVE
\hint{\refsto} {Strengthen post: $inv \land \neg guard \entails post(r)$}
\form{\ASSIGNMENT; \ENV:[inv, inv \land \neg guard]}
\ENDDERIVE
 % TODO: We could have ANDs if we want a sensible guard, or we have a ORs to match what the program has}
% Here be dragons, don't forgot to flip both of the signs if we need to

where $guard$ is a function that takes $i$, $j$, $k$ as implicit parameters, s.t.

\begin{eqnarray*}
guard(i, j, k) & \triangleq & (i \neq A.len\, \lor\, j \neq B.len\, \lor\, k \neq C.len)
\end{eqnarray*}

$\therefore$

\begin{eqnarray*}
inv \land \neg guard & \equiv & inv \land (i = A.len\, \land\, j = B.len\, \land k = C.len)
\end{eqnarray*}

Assuming $(i = A.len\, \land\, j = B.len\, \land k = C.len)$ holds, we can show that still\\ \[inv \land (i = A.len\, \land\, j = B.len\, \land k = C.len) ~\entails~ post(r)\]

$\because$

\begin{eqnarray*}
inv \land (i = A.len\, \land\, j = B.len\, \land k = C.len) & \equiv & inv[i, j, k \backslash A.len, B.len, C.len]\\
& \equiv & \INTERSECTION{r}{A.len}{B.len}{C.len}\\
    && \land~ \BOUNDED{r}{A.len}{B.len}{C.len}\\
& \equiv & (\POST) ~\land~ (r \in [0, D.len] ~\land~ \true ~\land~ \true ~\land~ \true)\\
& \equiv & (\POST) ~\land~ (r \in [0, D.len])\\
\end{eqnarray*}

\begin{eqnarray*}
(\POST) ~\land~ (r \in [0, D.len]) &~\entails~& post(r)\\
                                   &~\entails~& \POST
\end{eqnarray*}

\DERIVE
\hint{\refsto} {Repetition}
\form{
\begin{array}{l}
\ASSIGNMENT;\\
\Do~ \DOGUARD{i}{j}{k} \rightarrow\\
    \t1 \ENV:[inv\, \land\, guard,~ inv\, \land\, \VARIANTEXP]\\
\Od\\
\end{array}}
\ENDDERIVE

where

\begin{eqnarray*}
V & \triangleq & (A.len - i) + (B.len - j) + (C.len - k)\\
  & \triangleq & (A.len + B.len + C.len) - (i + j + k)
\end{eqnarray*}


\DERIVE
\hint{\doesntrefsto} {Selection: $inv\, \land\, guard \not\entails (G_1(i, j) \lor G_2(j, k) \lor G_3(k, i) \lor G_4(i, j, k))$}
\form{
\begin{array}{l}
\ASSIGNMENT;\\
\Do~ \DOGUARD{i}{j}{k} \rightarrow\\
    \t1 \begin{array}{l}
        \If~    \IFCASE{(A_i > B_j)}\\
        \Choice~ \IFCASE{(B_j > C_k)}\\
        \Choice~ \IFCASE{(C_k > A_i)}\\
        \Fi~     \IFCASE{(A_i = B_j) \land (B_j = C_k)}\\
    \end{array}\\
\Od\\
\end{array}}
\ENDDERIVE

where

\begin{eqnarray*}
G_1(i, j) & \triangleq & A_i > B_j\\
G_2(j, k) & \triangleq & B_j > C_k\\
G_3(k, i) & \triangleq & C_k > A_i\\
G_4(i, j, k) & \triangleq & (A_i = B_j) \land (B_j = C_k)\\
G_{union}(i, j, k) & \triangleq & (G_1(i, j) \lor G_2(j, k) \lor G_3(k, i) \lor G_4(i, j, k)\\
\end{eqnarray*}

$\therefore$

\DERIVE
\form{G_1(i, j) \lor G_2(j, k) \lor G_3(k, i) \lor G_4(i, j, k)}
\hint {\equiv} {Expansion of the guard definitions}
\form{(A_i > B_j) \lor (B_j > C_k) \lor (C_k > A_i) \lor ((A_i = B_j) \land (B_j = C_k))} 
\hint{\equiv} {Transitivity}
\form{\IFGUARD} 
\hint{\equiv} {Redefinition}
\form{G_{union}(i, j, k)}
\ENDDERIVE

Actions that have undefined behaviour, such as out-of-bounds array indexing, are inexpressible in the Guarded Query Language. Therefore, for any array-index pairing $A_i$, there is an implicit constraint that $i \in [0, A.len)$.\\

Thus

\DERIVE
\hint {\equiv} {Implicit constraints}
\form{(\IFGUARD) ~\land~ (\BOUNDEDNONINCLUSIVE{i}{j}{k})}
\ENDDERIVE

With these new implicit constraints, we can conclude that

\begin{eqnarray*}
inv \land guard  & \not\entails & (\IFGUARD)\\
    && \land~ (\BOUNDEDNONINCLUSIVE{i}{j}{k})
\end{eqnarray*}

For convenience

\begin{eqnarray*}
implicit\_constraints(i, j, k) & \triangleq & \BOUNDEDNONINCLUSIVE{i}{j}{k}
\end{eqnarray*}


By counter-example, let $i, j, k, r = A.len, 0, 0, 0$, and choosing $A = \{0\}$, $B = \{1\}$, $C = \{1\}$, s.t. $A.len = 1$, $B.len = 1$ and $C.len = 1$.

\begin{eqnarray*}
inv\SUBSTITUTION & \equiv & \INTERSECTION{0}{A.len}{0}{0}\\
    && \land \BOUNDED{0}{A.len}{0}{0}\\
    & \equiv & \emptyset = (A\, \cap\, \emptyset\, \cap\, \emptyset) ~\land~ (\true ~\land~ \true ~\land~ \true ~\land~ \true)\\
    & \equiv & \emptyset = \emptyset ~\land~ \true\\
    & \equiv & \true
\end{eqnarray*} 

\begin{eqnarray*}
guard\SUBSTITUTION & \equiv & \DOGUARD{i}{j}{k}\SUBSTITUTION\\
    & \equiv & \DOGUARD{A.len}{0}{0}\\
    & \equiv & \false\, \lor\, \true\, \lor\, \true\\
    & \equiv & \true
\end{eqnarray*} 

\begin{eqnarray*}
&& (G_{union}(i, j, k) ~\land~ implicit\_constraints(i, j, k))\SUBSTITUTION\\
& \equiv &\\
&& G_{union}(A.len, 0, 0) ~\land~ implicit\_constraints(A.len, 0, 0)\\
& \equiv &\\
&& G_{union}(A.len, 0, 0) ~\land~ (\BOUNDEDNONINCLUSIVE{A.len}{0}{0})\\
& \equiv &\\
&& G_{union}(A.len, 0, 0) ~\land~ (\false ~\land~ \true ~\land~ \true)\\
& \equiv &\\
&& G_{union}(A.len, 0, 0) ~\land~ \false\\
& \equiv &\\
&& \false
\end{eqnarray*}

$\therefore$

\begin{eqnarray*}
(inv\, \land\, guard)\SUBSTITUTION &\not\implies& (G_{union}(i, j, k)\\
&& \land~ implicit\_constraints(i, j, k))\SUBSTITUTION\\
inv\SUBSTITUTION\, \land\, guard\SUBSTITUTION &\not\implies& \false\\
\true\, \land\, \true &\not\implies& \false\\
\true &\not\implies& \false\\
\end{eqnarray*}

\textbf{The specification does not refine to the provided program.}\\

Instead of $\DOGUARD{i}{j}{k}$ as the $\Do$ guard, the choice of\\ $(i \neq A.len\, \land\, j \neq B.len\, \land\, k \neq C.len)$ prevents out-of-bounds array indexing in the guards of the $\If$ statement, and could be refined to by the specification. 

We can quickly show that this new guard, $guard'$, doesn't fall to the counter-example we used before.

\begin{eqnarray*}
guard'\SUBSTITUTION & \equiv & (i \neq A.len\, \land\, j \neq B.len\, \land\, k \neq C.len)\SUBSTITUTION\\
    & \equiv & (A.len \neq A.len\, \land\, 0 \neq B.len\, \land\, 0 \neq C.len)\\
    & \equiv & \false\, \land\, \true\, \land\, \true\\
    & \equiv & \false
\end{eqnarray*} 

s.t.

\begin{eqnarray*}
(inv\, \land\, guard')\SUBSTITUTION &\implies& (G_{union}(i, j, k)\\
&& \land~ implicit\_constraints(i, j, k))\SUBSTITUTION\\
inv\SUBSTITUTION\, \land\, guard'\SUBSTITUTION &\implies& \false\\
\true\, \land\, \false &\implies& \false\\
\false &\implies& \false
\end{eqnarray*}

\end{document}
